\documentclass{article}
\usepackage[utf8]{inputenc}
\usepackage{verbatim}
\usepackage{amsmath}
\usepackage{enumerate}
\usepackage{amsfonts}
\usepackage{microtype}
\usepackage{todonotes}
\usepackage[english]{babel}
\usepackage{graphicx}
\usepackage{tikz}
\usepackage{listings}
\usepackage{color}

\title{dRegAut afl 4}
\author{Hanus Rindom 201303613\\
        Christian Høj 201303530}
\date{May 2014}

\begin{document}

\maketitle

\section*{[CLRS] Exercise 26.2-11}
For any two vertices $ u $ and $ \nu $ in $ G $, we define a flow network $ G_{u \nu} $ consisting of the directed version of $ G $.  All edge capacities is set to 1, so that $ s = u $, and $ t = v $.\\
As required $ G_{u \nu}  $ has $ O(V) $ vertices and $ O(E) $ edges. We want all capacities to be $ 1 $, which have the effect that the number of edges crossing a cut is equal to the the capacity of the cut.

We assume $ f_{u \nu} $ denotes a maximum flow in $ G_{u \nu} $.
For any $ u \in V $, the edge connectivity $ k $ equals $ \min\limits_{ \nu \in V-\{u\}}  |f_{u \nu }| $.\\
If our assumtion holds, we can find k thus:

\lstset{frame=tb,
  language=Java,
  aboveskip=3mm,
  belowskip=3mm,
  showstringspaces=false,
  columns=flexible,
  basicstyle={\small\ttfamily},
  numbers=none,
  numberstyle=\tiny\color{gray},
  keywordstyle=\color{black},
  commentstyle=\color{dkgreen},
  stringstyle=\color{black},
  breaklines=true,
  breakatwhitespace=true
  tabsize=3
}

\definecolor{dkgreen}{rgb}{0,0.6,0}
\definecolor{gray}{rgb}{0.5,0.5,0.5}
\definecolor{mauve}{rgb}{0.58,0,0.82}


\begin{lstlisting}[mathescape]
EDGE-CONNECTIVITY(G)
select any vertex $ u \in  V $
for each $  \nu  \in V - \{ u \} $
    do set up the flow network $ G_{u \nu } $
        find the maximum flow $ f_{u \nu } $ on $ G_{u \nu } $
return the minimum of the $ |V| $ //: 1 max flow valuen: $ \min\limits_{ \nu \in V-\{u\}}  |f_{u \nu }| $

\end{lstlisting}


The assumtion follows theorem 26.6\footnote{Max-flow min-cut theorem}

\end{document}
