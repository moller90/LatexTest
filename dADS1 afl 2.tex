\documentclass{article}
\usepackage[utf8]{inputenc}
\usepackage{verbatim}
\usepackage{amsmath}
\title{dADS. Aflevering 2}
\author{
	Hanus Ejdesgaard Møller 201303613 \\
	Andreas Fogh-Pedersen 201303562 \\
	Christian Høj 201303530
}

\begin{document}
\maketitle
\section{Problem 3-4}
Versificer eller falsificer f(n) og g(n) som to positive asymptote funktioner.
	\subsection{A}
		\[ f(n)=O(g(n)) \; \text{implies} \; g(n)=O(f(n)) \]
			Vi vil gerne falsificere formodningen ved at definere funktioner hvor formodningen ikke er gældende.\\
			Hvid det antages at denne formodning er rigtig vil vi kunne falsificere formodningen ved at finde en værdi for $ c $ og $ n_0 $ hvor formodningen ikke gælder.
				
		Hvis vi definere $ f(n)=n $ og $ g(n)=n^2 $ vil de forskellige formodninger kunne illustreres således:
		
			\[ 0 \; \leq \; n \; \leq \; c*n^2 \text{ implies } 0 \; \leq \; n^2 \; \leq \; c*n \]
			
		Hvis c nu defineres til at være en vilkårlig konstant og n er en variable gående om uendelig, vil $ n^2 $ på et eller andet tidspunkt overstige $ c*n $, uanset hvad c er, og derfor er
		\[ f(n)=O(g(n)) \; \text{implies} \; g(n)=O(f(n)) \]
		ugyldigt.
				
	\subsection{B}
		\[ f(n)+g(n)=\Theta (g(n)) \: \text{implies} \: \Theta(min(f(n),g(n))) \]
		\[ h(n)=(min(f(n),g(n)) \]
		Vi vil gerne falsificere formodningen ved at definere funktioner hvor formodningen ikke er gældende.\\
		Formodningen kan formuleres således:
				\[ c_1*h(n) \leq g(n)+f(n) \leq c_2*h(n) \]
		\[ \text{Da } f(n)=n \text{ og } g(n)=n^2  \text{ kan formodningen udtrykkes således: } \] 
		\[ c_1*n \leq n+n^2 \leq c_2*n  \]
			Sætningen er falsk da $ n+n^2 $ på et eller andet tidspunkt overstige $ c_2+n $ for n voksende mod uendelig.
			
	\subsection{C}
        \[ f(n) = O(g(n))  \: \text{implies} \: log(f(n)) = O(log(g(n))) \] For $ log(g(n)) \geq 1 $ og $ f(n) \geq 1 $ for tilpas store n.\\
            Vi vil gerne verificere denne formodning ved at vise at første halvdel forudsætter anden halvdel af formodningen:
        \[ f(n) \leq c_1*g(n) \ \: \text{implies} \: \log(f(n)) ) \leq c_2* log(g(n))) \]
            Logaritmen tages på begge sider af første halvdel af formodningen:
        \[ log(f(n)) \leq c_2*log(g(n)) \]
            Logaritmeregneregler anvendes:
        \[ log(f(n)) \leq log(c_1)+log(g(n)) \]
            De er nu lig hinanden på nær $ log(c_1) \text{ og } c_2 $ Hvis vi kan få formuleret en sætning der definere hvad konstanten $ c_2 $ skal være, forudsætter første halvdel anden halvdel og formodningen er derved sand.\\
            Denne del af formodningen; $ log(c_1)+log(g(n)) $ skal være mindre eller lig med denne del af formodningen; $ c_2*log(g(n)) $, for at finde mindsteværdien for $ c_2 $\\
            Formodningen kan nu formuleres:
        \[ log(c_1)+log(g(n)) \leq c_2*log(g(n)) \]
            Der divideres med log(g(n)) på hver side:
        \[ \frac{log(c_1)}{log(g(n))}+1 \leq c_2 \]
            Da n er en variabel og $ log(g(n)) > 1 $ skal vi igen omformulere formodningen til:
        \[ \frac{log(c_1)}{log(g(n))}+1 \leq log(c_1)+1 \leq c_2 \]
            Det vil sige formodningen 
        \[ f(n) \leq c_1*g(n) \ \: \text{implies} \: \log(f(n)) ) \leq c_2* log(g(n))) \]
            er sand så længe $ c_2 \geq log(c_1)+1$
				
	\subsection{D}
		\[ f(n)=O(g(n)) \: \text{implies} \: 2^{f(n)}=O(2^{g(n)}) \]
		
		    Vi vil gerne falcificere denne formodning ved at vise at der findes funktioner som ikke forudsætte det påståede.\\
		        \[ f(n) \leq c*(g(n)) \: \text{implies} \: 2^{f(n)} \leq c*2^{g(n)} \]
		    Hvis $ f(n) = 2n $ og $ g(n)=n $ fås:
		        \[ 2n \leq c*n \: \text{implies} \: 2^{2n} \leq c*2^n \]
		    Ved isolation af c på højre side fås:
                \[ 2n \leq c*n \text{ implies } 2^n \leq c \]
            For en vilkårlig konstant c og n gående mod uendelig, vil $ 2^{n} $ efter n er blevet tilpas stor altid overstige en hvilken som helst konstant $ c $ og formodningen er derved falsk.
		
	\subsection{E}
		\[ f(n)=O((f(n))^2) \]
		Vi vil gerne falsificere formodningen ved at definere funktioner hvor formodningen ikke er gældende.\\
            \[ f(n) \leq c*((f(n))^2) \]
            \[ \text{Hvis  } f(n) = n \]
            \[ \text{c isoleres: } n \leq c*n^2 \Leftrightarrow \frac{1}{n} \leq c\]
                Hvis $ f(n_0) < 1 $ for n gående mod uendeligt, vil $ f(n) \; < \; f(n)^2 $ og formodningen er derved sand.\\
            \[ \text{Hvis  } f(n) = \frac{1}{n} \]  
            \[ \text{c isoleres: } \frac{1}{n} \leq \left( \frac{c}{n^2} \right) \Leftrightarrow n \leq c \]
        For $ c=1 \text{ og } n_0 = 1  $ vil formodningen ikke passe da n gående mod uendelig vil overstige konstanten c.\\
        Denne formodning vil være falsk for alle $ f(n) $ med resultatet mellem 0 og 1, og sand for alle reelle tal over 1.
\end{document}