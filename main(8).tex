\documentclass{article}
\usepackage[utf8]{inputenc}
\usepackage{amsmath}
\usepackage{enumerate}
\usepackage{microtype}
\usepackage{todonotes}

\title{dRegAut af 1 gen induktionsskridt}
\author{Hanus Rindom}
\date{May 2014}

\begin{document}

\maketitle

\subsection*{Induction step part 1 (+)}
    We want to prove $ Prefix(L(r_1+r_2))=L(r') $ where r is the regular expression $ r = r_1 + r_2 $ and $ r' = r'_1 + r'_2 $ where $ r'_1 \text{ and } r'_2 $ is given by the induction hypothesis.
    $$ L(r'_1)=Prefix(L(r_1)) \text{ and } L(r'_2)=Prefix(L(r_2)) $$
$ L(r') $ is pr definiton regular and therefore is r' also regular: 
    $$ L(r')=Prefix(L(r_1)) \cup Prefix(L(r_2)) $$
    \begin{equation*}
        \begin{split}
            Prefix(L(r)) &= Prefix(L(r_1 + r_2)) \\
            &= Prefix(L(r_1) \cup L(r_2)) \\
            &= \bigcup \{prefix(x) \text{ } | \text{ } x \in L(r_1) \cup L(r_2) \} \\
            &= \bigcup \{(prefix(x) \text{ } | \text{ } x \in L(r_1) \cup prefix(x) | x \in L(r_2)) \} \\
            &= \bigcup \{prefix(x) \text{ } | \text{ } x \in L(r_1) \cup \bigcup prefix(x) | x \in L(r_2) \} \\
            &= Prefix(L(r_1)) \cup Prefix(L(r_2))\\
            &= L(r'_1) \cup L(r'_2) \\
            &= L(r'_1 + r'_2) \\
            &= L(r')
        \end{split}
    \end{equation*}
        
The first part is now proven: $ Prefix(L(r)=L(r') $ 


\subsection*{Induction step part 2 (concatenating)}
        $$ \text{For } r=r_{1}r_2 \text{:\: } r' \text{ can be expressed: } r'=r'_1+r_{1}r'_2 $$
    
    Because a union is closed for prefix languages and concatering is closed for two regular languages. We can use Lemma 1 in our proof:\\
        $$ \forall x,y\in\Sigma^*: prefix(xy) = prefix(x) \cup \{x\}prefix(y)$$
    \begin{equation*}
        \begin{split}
        Prefix(L(r)) &= Prefix(L(r_1r_2)) \\
            &= Prefix(L(r_1)L(r_2)) \\
            &= \bigcup \{prefix(x) \text{ } | \text{ } x \in L(r_1)L(r_2)\} \\
            &= \bigcup \{prefix(xy) \text{ } | \text{ } x \in L(r_1) \text{ and } y \in (L(r_2)\} \\
            &= \bigcup \{prefix(x) \cup x \cdot prefix(y) \text{ } | \text{ } x \in L(r_1) \text{ and } y \in (L(r_2)\} \\
            &= \bigcup \{prefix(x) \text{ } | \text{ } x \in L(r_1) \} \cup \{x \cdot prefix(y) \text{ } | \text{ } x \in L(r_1) \text{ and } y \in (L(r_2)\} \\
            &= Prefix(L(r_1)) \cup x \cdot Prefix(L(r_2)) \text{ } | \text{ } x \in L(r_1) \\
            &= L(r'_1) \cup x \cdot L(r'_2) \text{ } | \text{ } x \in L(r'_1) \\
            &= L(r'_1 + x \cdot r'_2) \text{ } | \text{ } x \in L(r'_1) 
        \end{split}
    \end{equation*}
    
    The second part is now proven: $ Prefix(L(r)=L(r') $ with the help of Lemma 1.


\subsection*{Induction step part 3 *}
$ r=r_{1}^{*} \text{, } r'=r_{1}^{'*}  $. 
    Because we know concatering for prefix languages is closed, can we use Lemma 2 to prove $ r{'} $ regular:
        $$\forall i>0 \wedge S \subseteq \Sigma^*: prefix(S^i) = \cup_{k = 0...i-1} S^kprefix(S)$$
       
    \begin{align*}
        Prefix(L(r)) &= Prefix(L(r_1^*)) \\
            &= Prefix(L(r_1)^*) \\
            &= Prefix \left( \bigcup^{\infty}_{i=0}L(r_1)^i \right) \\
            &= \bigcup\{prefix(x) \text{ } | \text{ } x \in \bigcup^{\infty}_{i=0}L(_1)^i)\} \\
            &= \bigcup^{\infty}_{i=0} \bigcup\{prefix(x) \text{ } | \text{ } x \in L(_1)^i\} \\
            &= \left( \bigcup^{\infty}_{i=1}Prefix(L(r_1)^i)) \cup Prefix(L(r_1)^0 \right) \\
            &= \left( \bigcup^{\infty}_{i=1}\bigcup^{i-1}_{k=0}L(r_1)^kPrefix(L(r_q)) \cup Prefix(L(r_1))^0 \right)\\
            &= \left( \bigcup^{\infty}_{k=0}L(r_1)^kPrefix(L(r_1))\right) \cup \Lambda\\
            &= L(r')
    \end{align*}
The third part is now proven: $ Prefix(L(r)=L(r') $ with the help of part two and Lemma 2.
\end{document}