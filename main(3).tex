\documentclass{article}
\usepackage[utf8]{inputenc}
\usepackage{verbatim}
\usepackage{amsmath}
\usepackage{enumerate}
\usepackage{amsfonts}
\usepackage{microtype}
\usepackage{todonotes}

\title{dProgSprog afl 2}
\author{Hanus E. Rindom 201303613}
\date{April 2014}

\begin{document}

\maketitle

\section*{Exercise 1}
\begin{verbatim}
(seq (atom 1) (seq (atom 2) (seq (atom 3) (seq (atom 4) (empty)))))

          <regexp>
             |
             |
 (seq <regexp><regexp>)
        /          \
       /            \
      /              \
  (atom 1)    (seq <regexp><regexp>)
                      /          \
                     /            \
                    /              \
               (atom 2)  (seq <regexp><regexp>)
                                /        \
                               /          \
                              /            \
                        (atom 3)  (seq <regexp><regexp>)
                                         /          \
                                        /            \
                                       /              \
                                 (atom 4)           (empty)
Match (1 2 3 4)
\end{verbatim}
\newpage
\begin{verbatim}
(seq (atom 1) (seq (atom 2) (seq (atom 3) (atom 4))))

          <regexp>
             |
             |
 (seq <regexp><regexp>)
        /          \
       /            \
      /              \
  (atom 1)    (seq <regexp><regexp>)
                      /          \
                     /            \
                    /              \
               (atom 2)   (seq <regexp><regexp>)
                                /        \
                               /          \
                              /            \
                        (atom 3)         (atom 4)
Match (1 2 3 4)

(seq (seq (empty) (seq (atom 1) (atom 2))) (seq (empty) (seq (atom 3) (atom 4))))

                              <regexp>
                                 |
                                 |
                     (seq <regexp><regexp>)
                            /          \
                           /            \
                          /              \
       (seq <regexp><regexp>)          (seq <regexp><regexp>)
        /      \                           /          \
       /        \                         /            \
      /          \                       /              \
 (empty)    (seq <regexp><regexp>)    (empty)      (seq <regexp><regexp>)
                /      \                                /      \  
               /        \                              /        \  
              /          \                            /          \  
           (atom 1)     (atom 2)                  (atom 3)     (atom 4)
Matched (1 2 3 4)
\end{verbatim}
\newpage
\begin{verbatim}

(seq (seq (seq (atom 1) (atom 2)) (atom 3)) (atom 4))
 
                              <regexp>
                                 |
                                 |
                     (seq <regexp><regexp>)
                            /          \
                           /            \
                          /              \
    (seq <regexp><regexp>)            (seq <regexp><regexp>)
          /        \                          /          \
         /          \                        /            \
        /            \                      /              \
    (empty)  (seq <regexp><regexp>)     (empty)        (seq <regexp><regexp>)
                    /          \                              /          \
                   /            \                            /            \
              (atom 1)       (atom 2)                    (atom 3)       (atom 4)
Matched (1 2 3 4)

(seq (seq (seq (seq (empty) (atom 1)) (atom 2)) (atom 3)) (atom 4))
                               <regexp>
                                   |
                                   |
                       (seq <regexp><regexp>)
                         /              \
                        /                \
                       /                  \
               (seq <regexp><regexp>)    (atom 1)
                      /          \
                     /            \
                    /              \
       (seq <regexp><regexp>)     (atom 2)
                 /         \
                /           \
               /             \
  (seq <regexp><regexp>)   (atom 3)
            /       \
           /         \
          /           \
     (empty)         (atom1)
Matched (1 2 3 4)

\end{verbatim}
\section*{Exercise 6}
Vores induktionhypotese sider: 
    $$ \forall n1, n2 \text{ kan vi sige } n_1+n_2 =n_2+n_1 $$
    
Basistilfældet vil da være:
$$ 0 + n_2 = n_2 + 0 \leftrightarrow n_2=n_2 $$
Vi skal nu bevise at det gælder for k+1, vi laver induktion i $n_1$:
        \begin{equation*}
            \begin{split}
                (k+1) + n_2 &= 1+(k+n_2)\\
                &=(1+n_2) + k\\
                &= n_2+(k+1)\\
            \end{split}
        \end{equation*}

Dette viser at $ (k+1) + n_2 =  n_2+(k+1)$ hvilket er lig vores IH.\\
Vi har derfor vist at $ n_1+(1+n_2) = n_2 +(1+n_1)$ holder.

\section*{Exercise 7}

\subsection*{a}
Redifiner mul ved at tilgøje trace:
\begin{verbatim}
(define mul_traced
  (trace-lambda mul (n1 n2)
    (if (zero? n1)
      0
      (add n2 (mul_traced (1- n1) n2)))))

> (mul_traced 3 2)
|(mul_t 3 2)
| (mul_t 2 2)
| |(mul_t 1 2)
| | (mul_t 0 2)
| | 0
| |2
| 4
|6
6
\end{verbatim}


\subsection*{b}
Redifiner add ved at tilføje trace:
\begin{verbatim}
(define add_traced
  (trace-lambda add (n1 n2)
    (if (zero? n1)
      n2
      (1+ (add_traced (1- n1) n2)))))
\end{verbatim}
Redifiner mul ved at tilføje add$\_$trace:
\begin{verbatim}
(define mul_add_traced
  (lambda (n1 n2)
    (if (zero? n1)
        0
        (add_traced n2 (mul_add_traced (1- n1) n2)))))
        
> (mul_add_traced 2 3)
|(add_t 3 0)
| (add_t 2 0)
| |(add_t 1 0)
| | (add_t 0 0)
| | 0
| |1
| 2
|3
|(add_t 3 3)
| (add_t 2 3)
| |(add_t 1 3)
| | (add_t 0 3)
| | 3
| |4
| 5
|6
6
\end{verbatim}

\subsection*{3}
Vi skal nu vise at multiplikation er associativ:
$$ \forall x , y , z \in \mathbb{N} \text{ } : \text{ } (x * y) * z = x * (y * z) $$
Base case $z=0$:
    \begin{equation*}
        \begin{split}
            (x*y) * 0 &= 0\\
                &= x*0\\
                &= y*0\\
                &= x*(y*0)\\
        \end{split}
    \end{equation*}
Basecase holder da $ (x*y) * 0 = x*(y*0) $
\subsubsection*{Induktion hypotese}
    $$ (x*y)*k = x*(y*k) $$
\subsubsection*{Induktion skridt}
    Såfrem det gælder for k må det også gælde for k+1:
\begin{equation*}
    \begin{split}
        (x*y) * (1+k) &= ((x*y)*k + (x*y) \text{ pr definition} \\
            &= (x*(k*y)) + (x*y) \text{ pga. induktionshypotesen.} \\
            &= (x*y) + (x*(y*k)) \text{ bevist i exercise 6} \\
            &= x*(y+(y*k) \text{ Ved hjælp af distribution} \\
            &= x*((y*k)+y) \text{ bevist i exercise 6} \\
            &= x*(y*k)+(x*y) \text{ y sættes ind i parenteserne}\\
            &= x*(y*(k+1))  \text{ pr definition}\\
    \end{split}
\end{equation*}
Det holder og det er derved bevist at:
$$ (x*y)*z = x*(y*z) $$


\section*{Exercise 12}
Lav en rekursiv procedyre til at udregne fibonacci tal:
\begin{verbatim}
(define fib
  (lambda (n)
    (if (= n 0)
      0
      (if (= n 1)
        1
        (+ (fib (- n 1)) (fib (- n 2)))))))
\end{verbatim}


\section*{Exercise 23}
Definer en procedyre som kan tælle antallet af par i inputtet:
\begin{verbatim}
> (define number-of-pairs
    (trace-lambda pairs (v)
          (if (not (pair? v))
            0
            (+ 1 (number-of-pairs (car v))
              (number-of-pairs (cdr v))))))

> (number-of-pairs (cons (cons 10 20) 30))
|(pairs ((10 . 20) . 30))
| (pairs (10 . 20))
| |(pairs 10)
| |0
| |(pairs 20)
| |0
| 1
| (pairs 30)
| 0
|2
2

> (number-of-pairs 10)
|(pairs 10)
|0
0

> (number-of-pairs number-of-pairs)
|(pairs #<procedure>)
|0
0

> (number-of-pairs (list 1 2 3))
|(pairs (1 2 3))
| (pairs 1)
| 0
| (pairs (2 3))
| |(pairs 2)
| |0
| |(pairs (3))
| | (pairs 3)
| | 0
| | (pairs ())
| | 0
| |1
| 2
|3
3
> 
\end{verbatim}

\end{document}
