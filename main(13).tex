\documentclass{article}
\usepackage[utf8]{inputenc}
\usepackage{verbatim}
\usepackage{amsmath}
\usepackage{enumerate}
\usepackage{amsfonts}
\usepackage{microtype}
\usepackage{todonotes}
\usepackage[english]{babel}
\usepackage{graphicx}
\usepackage{tikz}
\usepackage{listings}
\usepackage{color}

\title{dRegAut afl 4}
\author{Hanus Rindom 201303613}
\date{May 2014}

\begin{document}

\maketitle

\section*{Martin 2.22 (b)}
In this exercise we will use the pumping lemma to show the following language cannot be accepted by an FA:
    $$ L= \{ a^i b^j a^k | k > i + j \} $$
For this we will use proof by contradiction.\\
Suppose for the sake of contradiction that L can be described by an FA with n states. We let $ x = a^n b a^{n+2} $ thus $ x \in L $.\footnote{We cannot choose the last a to be $ a^{n+1} $ because it would not hold the $ k > j+i $ condition.}\\
Then, because L can be described by an FA, and have to satisfy:
    \begin{equation*} \label{eq:Conditions}
    \begin{align}
        &\text{}\exists n > 0: \\
        &\text{\; \; \; }\forall x \in L \text{ hvor } |x| \geq n: \\
        &\text{\; \; \; \; \; \; }\exists u, v, w \in \Sigma *: \\
        &\text{\; \; \; \; \; \; \; \; \; }x=uvw \wedge |uv| \leq n \wedge |v| > 0 \wedge \forall m > 0: \\
        &\text{\; \; \; \; \; \; \; \; \; \; \; \; }uv^m w \in L: \\
    \end{align}
\end{equation}
Because we have chosen the first $ n $ symbols of $ x $ to be a, and $ |uv| \leq n \wedge |v| > 0 $ we can conclude that the string $ v $ must be $ a^j $ for some $ j > 0 $ and therefore must $ u v^2 w = a^{n+j} b a^{n+2} $.\\
Which is an contradiction: 
    $$ u v^2 w \in L \text{ but } n+j+1	\not< n+2 $$
The Pumping lemma was used to prove how the language $ L= \{ a^i b^j a^k | k > i + j \} $ could not be accepted by an FA.
\end{document}
