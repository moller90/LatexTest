\documentclass{article}
\usepackage[utf8]{inputenc}
\usepackage{verbatim}
\usepackage{amsmath}
\usepackage{enumerate}
\usepackage{amsfonts}
\usepackage{microtype}
\usepackage{todonotes}
\usepackage[english]{babel}
\usepackage{graphicx}
\usepackage{tikz}
\usepackage{listings}
\usepackage{color}

\title{dRegAut afl 4}
\author{Hanus Rindom 201303613}
\date{May 2014}

\begin{document}

\maketitle

\section*{Exercise 1}
In exercise 1 we are asked to trace the letrec-declared lambda abstraction.\\
The traced "add$\_$revisited" and "add$\_$revisited$\_$alt":

\begin{verbatim}
> (add_revisited-traced 3 3)
|(addTrace 3 3)
|(LetrecTrace 3 3)
| (LetrecTrace 2 3)
| |(LetrecTrace 1 3)
| | (LetrecTrace 0 3)
| | 3
| |4
| 5
|6
6

> (add_revisited_alt-traced 3 3)
|(AddAltTrace 3 3)
|(LetrecTrace 3)
| (LetrecTrace 2)
| |(LetrecTrace 1)
| | (LetrecTrace 0)
| | 3
| |4
| 5
|6
6
\end{verbatim}

Furthermore we are asked to program mul, pow and fac using local recursive procedures. Which we have some running exampels:

\begin{verbatim}
> (mul-traced_re 2 3)
|(mulTrace 2 3)
| (mulTrace 1 3)
| |(mulTrace 0 3)
| |0
| 3
|6
6

> (fac-traced_re 3)
|(facTrace 3)
| (facTrace 2)
| |(facTrace 1)
| | (facTrace 0)
| | 1
| |(mulTrace 1 1)
| | (mulTrace 0 1)
| | 0
| |1
| (mulTrace 2 1)
| |(mulTrace 1 1)
| | (mulTrace 0 1)
| | 0
| |1
| 2
|(mulTrace 3 2)
| (mulTrace 2 2)
| |(mulTrace 1 2)
| | (mulTrace 0 2)
| | 0
| |2
| 4
|6
6

> (pow-traced_re 3 3)
|(powTrace 3 3)
| (powTrace 3 2)
| |(powTrace 3 1)
| | (powTrace 3 0)
| | 1
| |(mulTrace 3 1)
| | (mulTrace 2 1)
| | |(mulTrace 1 1)
| | | (mulTrace 0 1)
| | | 0
| | |1
| | 2
| |3
| (mulTrace 3 3)
| |(mulTrace 2 3)
| | (mulTrace 1 3)
| | |(mulTrace 0 3)
| | |0
| | 3
| |6
| 9
|(mulTrace 3 9)
| (mulTrace 2 9)
| |(mulTrace 1 9)
| | (mulTrace 0 9)
| | 0
| |9
| 18
|27
27
\end{verbatim}


\section*{Exercise 5}
In this exercise, we are asked to implement three structurally
recursive procedure: proper-list-ref, proper-list-tail and proper-list-head. these respectively emulate the predefined procedures list-ref, list-head and list-tail:
\begin{verbatim}
> (proper-list-ref (list 0 1 2 3 4 5 6 7 8 9) 8)
8
> (proper-list-head (list 0 1 2 3 4 5 6 7 8 9) 2)
(0 1)
> (proper-list-tail (list 0 1 2 3 4 5 6 7 8 9) 6)
(6 7 8 9)
\end{verbatim}

\section*{Exercise 8}
In this exercise we will write an procedure "set-normalize" which will yeald an imput list but without dublications.\\
Exampels on the procedure set-normalize:
\begin{verbatim}
> (set-normalize (list 9 0 1 2 -1 3 4 2 5 3 3 3 4 5 6 7 8 9))
(0 1 -1 2 3 4 5 6 7 8 9)
\end{verbatim}

If we use integers we can have it sortet by the predefined procedure "sort":

\begin{verbatim}
> (sort < (set-normalize (list 8 6 2 5 4 7 -6 -8)))
> (-8 -6 2 4 5 6 7 8)
\end{verbatim}

\section*{Exercise 9}
In this exercise, we will implement "set-union", "set-intersection", and "set-difference" over representations of sets.
lists of elements without duplication:
\begin{verbatim}
> (set-union '(a b) '(b c))
(a b c)
> (set-intersection '(a b) '(b c))
(b)
> (set-difference '(a b) '(b c))
(a)
\end{verbatim}
I am not quite sure about what i am doing wrong, a hint could be helpful.
\section*{Exercise 20}
In this exercise we modify run-byte-code-program so it does not operate on numbers anymore but rather representations of numbers:

\begin{verbatim}
> (run-byte-code-program_Magritte '(byte-code-program ((PUSH 6))))
(literal 6)

> (run-byte-code-program_Magritte '(byte-code-program ((PUSH 40) (PUSH 2) (ADD))))
(plus (literal 40) (literal 2))

> (run-byte-code-program_Magritte '(byte-code-program ((PUSH 7) (PUSH 6) (MUL))))
(times (literal 7) (literal 6))
\end{verbatim}

Rather than calculation the numbers we get syntactic representations of calculations.

\end{document}