\documentclass{article}
\usepackage[utf8]{inputenc}
\usepackage{verbatim}
\title{dIntDes afl 3}
\author{
	Hanus Ejdesgaard Møller \\
	Andreas Fogh-Pedersen \\
	Christian Høj
}

\begin{document}
\maketitle
\section{Kritik og evaluering}
Projektet blev fremlagt for andre tilstedeværende grupper og feedback blev givet. Der var enighed om ideens anvendelighed og funktionalitet, samt danskernes evne til small-talk.\\
Langt det mest diskuterede omhandlede brugeroplevelsen, at vores produkt fra vores side var tænkt til cafegæster er nu noget vi bliver nød til at revurdere, hvilket er godt, da det var en problematik vi ikke selv havde set.

Den fysiske placering skal ændres fra cafeer til bar, Baristaer eller stripklub.\\
Vi vil igen se på og påpege produktets brugskontekst, det skal på ingen måde være til gene eller distraherende for brugeren og kun virke som et værktøj brugeren er klar over og kunne finde interessant at bruge.
\section{Scenarier}
Vi forestiller os et senarie på en bar, bordene har fået installeret vores ide og alt er fryd og gammen. Gæsterne kommer ind, deriblandt en gruppe venner, bestiller øl og sætter sig ved et bort. 
\section{Low fidelity prototype}

\section{Konceptuel model}

\section{Interaktionstyper}

\section{High fidelity prototyper}

\end{document}