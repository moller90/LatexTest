\documentclass{article}
\usepackage[utf8]{inputenc}
\usepackage{amsmath}
\usepackage{microtype}
\usepackage{todonotes}

\title{RegAut afl. 1 genaflevering}
\author{Anders Kostending, 201303537}
\date{May 2014}

\begin{document}

\maketitle

\section{Del 2 - induktionsskridt}
\subsubsection*{Del 1 med "+"}
    Vi starter med at bevise det ved "+".\\
    Vi tager et regulært udtryk $r = r_1 + r_2$\\
    Så sættes $r' = r'_1 + r'_2$ hvor $r'_1$ og $r'_2$ er givet ved induktionshypotesen så $L(r'_1) = Prefix(L(r_1)$.\\
    $L(r')$ er regulært, da $r'$ er et regulært udtryk og $L(r') = prefix(L(r'_1)) \cup Prefix(L(r'_2))$\\
    \\
    Hvis vi skal samle hvad det er der står så vil det være
    \begin{equation*}
        \begin{split}
            Prefix(L(r)) &= Prefix(L(r_1 + r_2)) \\
            &= Prefix(L(r_1) \cup L(r_2)) \\
            &= \bigcup{prefix(x) | x \in L(r_1) \cup L(r_2)} \\
            &= \bigcup(prefix(x) | x \in L(r_1) \cup prefix(x) | x \in L(r_2)) \\
            &= \bigcup prefix(x) | x \in L(r_1) \cup \bigcup prefix(x) | x \in L(r_2) \\
            &= Prefix(L(r_1)) \cup Prefix(L(r_2))\\
            &= L(r'_1) \cup L(r'_2) \\
            &= L(r'_1 + r'_2) \\
            &= L(r')
        \end{split}
    \end{equation*}
    
    
    Dette er nu bevist.

\newpage

\subsubsection*{Del 2 med "*"}
    Nu vil vi bevise det ved "*".\\
    Vi tager et regulært udtryk $r=r_1^*$\\
    Så sættes $r' = r'_1^*$ og som det vises i del 3, så er konkatenring ved prefix-sprogene bevist.
    Vi bruger derfor Lemma 2 som siger $$\forall i>0 \wedge S \subseteq \Sigma^*: prefix(S^i) = \cup_{k = 0...i-1} S^kprefix(S)$$
    som ved brug giver os
    
    \begin{align*}
        Prefix(L(r)) &= Prefix(L(r_1^*)) \\
            &= Prefix(L(r_1)^*) \\
            &= Prefix(\bigcup^{\infty}_{i=0}L(r_1)^i) \\
            &= \bigcup\{prefix(x) | x \in \bigcup^{\infty}_{i=0}L(_1)^i)\} \\
            &= \bigcup^{\infty}_{i=0} \bigcup\{prefix(x) | x \in L(_1)^i\} \\
            &= (\bigcup^{\infty}_{i=1}Prefix(L(r_1)^i)) \cup Prefix(L(r_1)^0) \\
            &= (\bigcup^{\infty}_{i=1}\bigcup^{i-1}_{k=0}L(r_1)^kPrefix(L(r_q)) \cup Prefix(L(r_1))^0)\\
            &= (\bigcup^{\infty}_{k=0}L(r_1)^kPrefix(L(r_1))) \cup \Lambda\\
            %&= \bigcup^{\infty}_{i=1} L(\cup_{k = 0...i-1}r_1^kprefix(r)) \cup Prefix(L(r_1)^0)\\
            &= L(r')
    \end{align*}
    
    Man kunne sætte en de to union sammen til en da de overlappede i union's.
    
    Dette er nu bevist ved hjælp af den givne Lemma

\end{document}
