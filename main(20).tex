\documentclass{article}
\usepackage[utf8]{inputenc}
\usepackage{verbatim}
\usepackage{amsmath}
\usepackage{alltt}

\usepackage{listings}
\usepackage{color}

\definecolor{dkgreen}{rgb}{0,0.6,0}
\definecolor{gray}{rgb}{0.5,0.5,0.5}
\definecolor{mauve}{rgb}{0.58,0,0.82}

\lstset{frame=tb,
  language=Java,
  aboveskip=3mm,
  belowskip=3mm,
  showstringspaces=false,
  columns=flexible,
  basicstyle={\small\ttfamily},
  numbers=none,
  numberstyle=\tiny\color{black},
  keywordstyle=\color{black},
  commentstyle=\color{dkgreen},
  stringstyle=\color{black},
  breaklines=true,
  breakatwhitespace=true
  tabsize=3
}

\title{dADS 2 afl 1}
\author{
	Hanus Ejdesgaard Møller 201303613 \\
	Christian Høj 201303530}

\begin{document}
\maketitle
\section*{opg 4 a}
We assume that a silhouette is composed of a list of numbers, that has individual x and h numbers that can be accessed by $x_i$ and $h_i$. $S(x_0, h_1, x_1, h_2, x_2, … h_n, x_n)$. h3 would as an example be the 6th number in the list. 

\begin{alltt}
1   Insert-House(S, l, h, r)
2   R = new empty siluet
3   For \(x\sb{0} -> x\sb{n}\)
We go throug all the x points in the silhouette from \(x\sb{0} to x\sb{n}\).
4       If \(x\sb{i}\)<l
5           Copy set \(S(x\sb{i},h\sb{i}+1)\) to R

We draws an new silhouette,and copy as much as we can form the
old one.

6       If \(x\sb{i} \leq l \leq x\sb{i}+1\)
7          If \(h\sb{i}+1<h\)
8              Insert \((l,h)\) to R

If necessary add the left point of the house to the new
silhouette.

9       If \(r > x\sb{i}>l\)
10          if \(h\sb{i}+1>h\)
11              Insert \((x\sb{i},h\sb{i}+1)\) to R
12          if \(h\sb{i}+1 < h\)
13              Insert \((x\sb{i} , h)\) to R

14      If \(x\sb{i} \leq r \leq x\sb{i}+1\)
15          If \(h\sb{i}+1<h\)
16              Insert \((r,h\sb{i}+1)\) to R  

If necessary add the Right point of the house to the new
silhouette.

17      If \(x_i r\)
18          Copy set \(S(x\sb{i} , h\sb{i}+1)\) to R

We continue the new silhouette, and copy as much as we can
form the old one.
\end{alltt}


\section*{opg 4 b}
By discussing the exercise in length with several other groups we have found a solution, which differ considerably from our previous work\footnote{in case you where wondering}\\
Given 2 arrays, each with an silhouette, it is our job to combine them.\\
First we copy the two arrays which we call $ S_{1new} $ and $ S_{2new} $.\\
We will include all the x-coordinates from $ S_{2} $ in $ S_{1new} $ and include all the x-coordinates from $ S_{1} $ in $ S_{2new} $.\\
We have now created two lists witch contains all values from the two original lists, and will use these for our merge algorithm.\\
Between these two silhouettes we want to add the highest values of h form the two arrays. This is done by changing ouy copy $ S_{1new} $ and compare it to $ S_{2new} $, thereafter we erase any double-values of height h in $ S_{1new} $ and in the end we return the array of $ S_{1new} $.

\begin{alltt}
MERGE(\(S\sb{1},\sb{2}\))
\(COPY(S\sb{1}. S\sb{1new})\)  Create new arrays to insert our given array into.
h=0
j=0
for \(i=0\) to n in \(S\sb{2}\)  copy existing array into new empty array, which we
                            can change as we please.
if \(S\sb{1new} [J] = S\sb{1new}.length-1 and S\sb{1} < S\sb{1}.length\)
                    In case we look at the last element in the list.
    insert \(0,S\sb{1}[i] after S\sb{2new}[j]\) 
    i=i+2             Count two up so we only count the height-values
    j=j+2
if i<j 
    insert \(S\sb{1}[i], h before S\sb{2new}[j]\)
    i=i+2
    j=j+2
else
    j=j+2
    h=j-1   A copy of \(\sb{1}\) is now createt

\(COPY(S\sb{2}. S\sb{2new})\) 
h=0
j=0

for \(i=0\) to n in \(S\sb{1}\)
if \(S\sb{2new}[j]=S\sb{2new}.length-1 and S\sb{1}[i] < S\sb{1}.length\)
    insert 0, \(S\sb{1}[i] after S\sb{2new}[j]\)
    i=i+2
    j=j+2
if i<j
    insert \(S\sb{1}[i]\), h before \(S\sb{2new}[j]\)
    i=i+2
    j=j+2
else
    j=j+2
    h=j-1    A copy of \(\sb{2}\) is now created

for \(i=1 to n in S\sb{1new}\)   We begin to change \(S\sb{1new}\)
    if \(S\sb{2new}[i] > S\sb{1new}[i]\)    check height difference.
            \(S\sb{1new}[i] = S\sb{2new}[i]\) Change height
        i=i+2
        
for i=3 to n is \(S\sb{1new}\)   
    if \(S\sb{1new}[i] > S\sb{1new}[i-2]\)  If height has not changed since last coordinate.
        remove \(S\sb{1new}[i]\)
        remove \(S\sb{1new}[i-1]\)
    i=i+2
return \(S\sb{1new}\)

\end{alltt}

\section*{opg 4 c}
We can use the divide and counquer algorithem for this problem.\\
By using MergeSort, which given an array, recursively dividing it until we only have one element in the list. We can then combine each element withevery other and end up with an complete silhuet.\\
This algorithem will have a runtime like MergeSort which is $O(n logn)$
\begin{lstlisting}[mathescape]
COMBINE-SILHUETTE(A, p, r)
if (p<r)
    q= $\lfloor$(p+r)/2$ \rfloor $
    S_1 = COMBINE-SILHUETTE(A, p, r)
    S_2 = COMBINE-SILHUETTE(A, p, r)
    MERGE(S1,S2)        As described in exercise b
\end{lstlisting}

All hail hypnotoad.

\end{document}