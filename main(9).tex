\documentclass{article}
\usepackage[utf8]{inputenc}
\usepackage{amsmath}
\usepackage{enumerate}
\usepackage{microtype}
\usepackage{todonotes}

\title{dRegAut afl 1}
\author{Hanus Ejdesgaard Rindom 201303613}
\date{April 2014}

\begin{document}

\maketitle

\section{Problem 1}
    Define a language S for $ \Sigma $, the Prefix(S) is a language consisting of prefixes of strings from S:
        $$ Prefix(S)=\cup Prefix(x) | x \in S= \cup_{x \in S} Prefix(x) $$
        
        $$ r=a+bc $$
    What is the language L(r) and Prefix(L(r)):\\
        $$ L(r)= \{a,bc\} $$
        $$ Prefix(L(r))= prefix(a) \cup prefix(bc) = \{\Lambda,a,b,bc\}$$
        
\section{Problem 2}
    In this section we will prove how any regular language is closed under Prefix: If S is an regular language, then Prefix(S) is an regular language.
    
    This section will take into account two Lemma:\\
    Lemma 1: $$ \forall x,y \in \Sigma^{*}: prefix(xy)=prefix(x) \cup \{x\}prefix(y) $$
    Lemma 2: $$ \forall i>0 \text{ and } S \subseteq \Sigma{*}: Prefix(S^i)= {\bigcup}_{k=0...i-1} S^k Prefix(S) $$
    
    The expression L(r)=S is regular due to the definition of a regular language.
    
    The existence of r' is to be proven correct for L(r')=Prefix(S), if this can be proved Prefix(S) is regular.
    
\subsection*{Basecase}
    In this step we will show how to use prefix for 2 basic expressions without loss    of regularity.\\
    The two expressions for r is Ø and a, where a is any symbol in the alphabet.\\
    The two basic expressions for r is :
        $$ r=Ø $$
        $$ r'=Ø: \text{ } L(r')=\{Ø\}=Prefix(\{Ø\})=\{Ø\}  $$
    
        $$ r=a \in E $$
        $$ r'=a \text{ } L(r')=\{a\}=Prefix(\{a\})=\{a\} $$
    
\subsection*{Induction Hypotheses}
    There exist a regular expression r' where L(r') = Prefix(L(r)).

\subsection*{Induction step part 1 (+)}
    We want to prove $ Prefix(L(r_1+r_2))=L(r') $ where r is the regular expression $ r = r_1 + r_2 $ and $ r' = r'_1 + r'_2 $ where $ r'_1 \text{ and } r'_2 $ is prefixes.
    $$ L(r'_1)=Prefix(L(r_1)) \text{ and } L(r'_2)=Prefix(L(r_2)) $$
$ L(r') $ will then be regular: 
    $$ L(r')=Prefix(L(r_1)) \cup Prefix(L(r_2)) $$

        \begin{equation*}
            \begin{split}
                Prefix(L(r)) &= Prefix(L(r_1+r_2))\\
                &= Prefix(L(r_1) \cup L(r_2)) \\
                &= Prefix(L(r_1))+Prefix(L(r_2)) \\
                &= \bigcup prefix(x) \text{ } | \text{ } x \in L(r_1) \cup L(r_2) \\
                &= L(r'_1) \cup L(r'_2) \\
                &= L(r'_1+r'_2) \\
                &= L(r') \\
            \end{split}
        \end{equation*}
        
The first part is now proven: $ Prefix(L(r)=L(r') $ 

\subsection*{Induction step part 2 (concatenating)}
        $$ \text{For } r=r_{1}r_2 \text{ } r' \text{ can be expressed: } r'=r'_1+r_{1}r'_2 $$
    
    Because a union is closed for prefix languages and concatering is closed for two regular languages. We can use therom 1 to prove $ r' $ to be regular:
        $$ \forall x,y\in\Sigma^*: prefix(xy) = prefix(x) \cup \{x\}prefix(y)$$
    \begin{equation*}
        \begin{split}
            Prefix(L(r)) &= Prefix(L(r_{1}r_2)) \\
            &= Prefix(L(r_1)) \cup L(r_1)Prefix(L(r_2)) \\
            &= L(r'_1 \cup r_{1}L(r'_2) \\
            &=L(r') \\
        \end{split}
    \end{equation*}
    
    The second part is now proven: $ Prefix(L(r)=L(r') $ with the help of Lemma 1.


\subsection*{Induction step part 3 *}
        $$ \text{For } r=r_{1}* \text{ } r' \text{ can be expressed: } r'= {\bigcup} _{k=0...i-1} r^k_1 prefix(r1) $$
    Because we know concatering for prefix languages is closed, can we use Lemma 2 to prove $ r{'} $ regular:
        $$ \forall i>0 \text{ and } S \subseteq \Sigma{^*}: Prefix(S^i)= {\bigcup}_{k=0...i-1} S^k Prefix(S) $$
        $$ prefix(L(r)) = prefix(L(r^*_1)) = L \left( {\bigcup} _{k=0...i-1}r^k_1 prefix(r)\right) = L(r') $$

The third part is now proven: $ Prefix(L(r)=L(r') $ with the help of part two and therom 2.
\end{document}