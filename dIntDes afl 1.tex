\documentclass{article}
\usepackage[utf8]{inputenc}
\usepackage{verbatim}
\title{dIntDes afl 1}
\author{
	Hanus Ejdesgaard Møller \\
	Andreas Fogh-Pedersen \\
	Christian Høj
}

\begin{document}
\maketitle
\section{Det valgte design case.}
    Af designcase har vi valgt $ i $, her har vi valgt at fokusere på interaktionen mellem cafégæster, og hvordan denne kan forbedres.\\
    Efter lang tid diskussion og case-forslag fandt vi frem til at danskere er rigtig dårlige til small-talk. I busser, tog og andre steder i det offentelige kommunikere vi danskere ikke med mindre vi bliver skuppet ud i det, toget eller bussen er forsinket og det er da også træls, det kan vi alle blive enige om.\\
    Cafeer har samme type af interaktion, en eller to mennesker sætter sig ned og nyder en kop kaffe e.l. men hvad hvis vi skabte en eller anden for for interaktion med en maskine som kunne hjælpe med at bryde isen mellem gæsterne.
    
\section{Undersøgelser af brugskontekst og brugernes behov}

    Det overordnede mål er at kortlægge om der overhovedet er behov for styrkelse af den sociale interaktion i en cafésituation, om behovet er småt eller måske ikkeeksisterende.

    Undersøgelsen forsøger at kortlægge hvorvidt produktet, der arbejdes med, er en del af brugerens behov. Formen er et spørgeskema, som kan besvares online. Brugeren ledes hen til spørgeskemaet ved hjælp af et link. Dette giver problemet, at man ikke kan kontrollere om folk er sandfærdige i deres svar omkring alder og køn. Til gengæld giver det brugeren mere anonymitet, hvilket giver forhåbningen om at folk er mindre tilbøjelige til at svare, hvad de tror vi gerne vil høre.

    Det nuværende produkt er stadig blot i tanken og i et så tidligt stadie, at en undersøgelse af brugerens behov vil lede til større forståelse af produktkrav, og hvordan produktet passer ind i hverdagen og den almindelige interaktion mennesker imellem. Der er en røkke fordele og ulemper i at benytte online spørgeskema hvilket  blandt andet kan dække:\\
\begin{tabular}{ | l | l | }
    \hline
    \textbf{Fordele} & \textbf{Ulemper} \\ \hline
    Nem måde at få generelle svar & Meget kvantitativ undersøgelse \\ \hline
    Nem måde at få mange besvarelser & Ikke specielt personligt \\
    \hline
\end{tabular}\\
\\
\\
\begin{tabular}{ | l | l | }
    \hline
    \textbf{Fordele (ved at det er online)} & \textbf{Ulemper (ved at det er online)} \\ \hline
    Mere anonymitet for brugeren & Mindre følelse af ansvar overfor forfatter \\ \hline
    Nemmere end at blive antastet på gaden & Teknisk udfordrede for uvante pc-brugere \\ \hline
    Svar der ikke er bundne af nervøsitet & Utryghed ved noget brugere ikke kan se \\ \hline
\end{tabular}\\
    
    Spørgeskemaer er en nem måde at få generelle svar, og da det ikke er rettet til bestemte mennesker (dog muligvis grupper af mennesker), kan det skrives meget generelt. Dette giver mulighed for at få mange besvarelser, hvilket også er grunden til, at den typiske spørgeskemaundersøgelse er en meget kvalitativ metode. At spørgeskemaer ikke er specielt personlige betyder, at nogle personer muligvis vil svare meget overfladisk. Dette er grunden til, at det omtalte spørgeskema er relativt kort - et tro på, at brugeren vil gå i dybden med det. At det er muligt at besvare spørgeskemaet online giver naturligvis et problem for teknisk udfordrede, der vil have sværere ved at besvare denne type frem for den fysiske slags. Disse brugere vil, alt efter hvor udfordrede de er, have tiltagende utryghed ved metoden. Det er et bevidst valg at spørgeskemaet er online:
\begin{itemize}

  \item Brugere kan besvare det i trygge rammer
  \item De bliver ikke antastet på gaden
  \item Det er en nem måde at ordne data (SurveyMonkey gør det meste)
  \item De fleste mennesker bruger jævnligt en computer

\end{itemize}

    I alt svarede 35 på vores skema. Det var primært mennesker i aldersgruppen 20-25, og størstedelen af dem var mænd.
    
    I besvarelserne observeres der er en splittelse i meningerne om produktet. Der er en tendens til at folk syntes om vores idé som en ekstra assistance til smalltalk, men som sådan ikke så meget interaktionen cafegæster imellem.
    
    Vi havde et frit felt til sidst i skemaet, hvor folk kunne kommentere hvis de havde lyst. Ca. 15 benyttede sig af dette og disse kan opdeles i 3 katogerier: De negative, som fandt ideen forstyrrende. De negativ-neutrale, som fandt ideen spændendene men frygtede samtidig at det ville gå ud over person til person kommunikation og til sidst de posektive, som var dem der fandt ideen tiltagende, det ser ud til at de så produktet som et slags ekstra feature, som man ikke nødvendigvis skulle bruge.
    
    Her er vedlagt et par af kommenterene fra sidste spørgsmål:
\begin{itemize}
    
    \item ” det skal være nem at benytte og skal ikke tage for meget opmærksomhed fra eventuelle samtale”
    \item ” hvis jeg er på cafe sidder jeg og snakker med mine venner og ville synes det var enormt irriterende at blive forstyrret.”
    \item ” Har ikke brug for en bordplade der kan hjælpe mig med at snakke”
    \item ” Det lyder som et godt projekt, hvor man hurtigt vil kunne danne venskaber med nye mennesker. ”
    
\end{itemize}
    
\subsection{Analyse af resultater}
    Af adspurgte har 42\% svært eller meget svært ved smalltalk. Dette tyder på, at der er marked for vores produkt. De 42\% kan man holde sammen med de 15\%, der kan bruge et hjælpemiddel til smalltalk. Dette virker som en modsætning, men 51\% mener, at de kan bruge det i nogle situationer. Det er disse "situationer" det endelige produkt skal ramme. For at få disse situationer kortlagt, kræves yderligere undersøgelser, hvilket stemmer overens med den iterative fremgangsmåde.
    
    Af adspurgte er 75\% interesserede eller meget interesserede i et redskab, der kan fremme interaktion mellem cafégæster, blot det ikke forstyrrer. Flertallet mener altså, at produktet skal tage meget lidt opmærksomhed. Dette vil naturligvis indgå i prototyper og fremtidige undersøgelser af brugernes behov.
    
    I de yderligere kommentarer giver en respondent udtryk for, at han/hun ikke har nok information om produktet. Dette skyldes flere ting: dels er det svært at forklare produktet, hvilken sammenhæng den skal indgå i osv. på få linjer. Der kan forsøges at få defineret en klar "definition" i gruppen, så alle har en fælles forestilling om, hvordan produktet skal se ud, fungere i sammenhængen osv, hvilket forhåbentlig munder ud i mere klare formuleringer i brugerundersøgelser.
    
    Der blev i flere kommentarer givet udtryk for, at det skulle være enkelt at bruge. Der vil tænkes over:
        \begin{itemize}
            \item Hvordan skal produktet indgå i "landskabet"?
            \item Hvordan skal selve layout se ud for at det forstyrrer brugeren mindst muligt samtidig med, at det er enkelt at bruge?
            
        \end{itemize}
    Flere respondenter giver udtryk for, at de ikke er interesserede i interaktion med cafégæster, de ikke kender på forhånd. Dette kan tackles ved, at brugeren får mulighed for at offentliggøre sin interaktion med produktet til andre enheder. Derved tages også hånd om eventuelle brugere, der vil have fuld kontrol over, hvad produktet gør, når lige netop de betjener det.
    
    En respondent giver udtryk for, at det muligvis kunne erstatte en nyligt ankommen tendens: nogle mennesker benytter deres smartphone når de er sammen med andre mennesker. Dette bliver, af mange, set som uhøfligt. Denne vinkel er ikke overvejet før. Naturligvis bestræbes der ikke på, at produktet skal kunne benyttes til at sende sms eller ringe, men hvis menneskers interaktion med smartphones skyldes mangel på aktivitet, kan produktet i nogle tilfælde fungere som en erstatning.
    
    Undersøgelsen giver indtrykket, at der i nogen grad er brug for produktet i den nuværende form, men for at gøre produktet mere generelt (det antages, at cafégæster findes i næsten alle demografier) er det vigtigt at situationen bliver tænkt med: cafégæster er der for at snakke med venner; ikke for at blive underholdt. Det er vigtigt, at produktet fungerer som et middel til social interaktion og ikke som en uundværlig ting. I yderligere undersøgelser vil der blive lagt vægt på at kortlægge, i hvilke situationer folk mener at produktet ville være nyttigt for dem.

\section{Etablering af krav}
    I forhold til usability kan vi se fra spørgeskemaet at det er vigtigt vi lægger stor vægt på ”Easy to learn”, således enhver der sætter sig med det samme forstår hvordan det virker, for at modvirker den gene der kan opstå for brugerne hvis de skal til at bakse rund med det..
    \begin{itemize}
    
        \item ”skal ikke tage for meget opmærksomhed fra eventuelle samtaler”
        \item ”føler at man har fuld kontrol over indholdet på bordet.”
            
    \end{itemize}
    Det er meget vigtigt at vores applikation ikke tager unødvendig opmærksomhed, og at det er en sjov ”flydende” oplevelse, så det er nemmere at lokke skeptikere til at bruge den. Vi så at over 40\% syntes bordet kunne være en god ide, så længe det ikke var et forstyrrende element. Vi vil derfor gå efter høj visibility, så vejen til at udnytte bordet er optimal. Desuden skal der være virkelig god feedback, med tydelige reaktioner på interaktion, så den tid man ser ned i bordet er minimeret. Bordet skal ikke i sig selv kræve noget af brugeren – vi ønsker netop vi at give stimulation til gode samtaler gæster imellem.

\section{Tidlige design ideer}
    Før vi kom frem til vor cafe-ide, havde to andre ideer. Hvoraf den mest diskuterede var det interaktive busstoppested. Her var ideen at en app kunne downloades og hver bus var GPS-tracked, appen skulle da være i stand til at kunne vise meget præcist, hvornår næste bus kommer.
    
    Denne ide byggede på at have høj effektivitet, være et praktisk værktøj\footnote{Ideen kom af at busplanerne ikke passede så godt mens sneen lå højt} og lav learnabillity.\\

    Den anden og hurtigt skrottede ide var et aktivt computerspil til børn, hvis formål skulle være at understøtte motion og bevægelse, denne ide kom af at en holdkammerat har et veludbygget netværk gennem et tidligere arbejde.
    
    Selv om vi aldrig kom med et decideret forslag var vi enige om at det vigtigste usability goal skulle være safe to use, nu da produktet ville være beregnet til børn. For børn er brugeroplevelsen også altafgørende, hvis det ikke er sjovt eller på anden måde dragende er hele produktet nyttesløs.
    
\section{Resultat af kritik}
	Efter at spørgeskemaet var færdig og evalueret i gruppen blev det meget hurtigt klart at den kunne være lavet mere præcist, hvilket en bruger direkte har været inde og skrive \emph{"jeg har ikke nok information om denne til at kunne forstå konceptet. er det en bordplade hvor man modtager informationer på eller er der en skærm?"}. Dette var dog i nogen grad et bevist valg da vi ikke ville give brugeren et meget konkret produkt at forholde sig til, så brugeren ikke var så bundet i deres besvarelser, hvilket heldigvis også har givet en tilpas varieret respons til at vi kan revurdere vores egne tanker og forventninger til produktet.
	
	Andet vurdering af produktet fik vi ved godkendelsen af projektet, hvor vi vendte de forskellige ideer og tanker om konceptet og derigennem opbyggedefundamentet for vores ide.
\end{document}